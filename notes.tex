
- Funktionsweise und Eigenschaften des EEG erklären
- Frequenzbereiche des EEG erkennen, benennen, und erklären
- erklären, wie sich das EEG im Schlaf verändert
- interne und externe Regulierung des zirkadianen Rhythmus erklären
- die Funktionen des Schlafes angeben
- die Schlafphasen benennen und beschreiben
- erklären, wie der Übergang zwischen Schlaf und Wachheit reguliert wird
- erklären, wie der Überang zwischen verschiedenen Phasen des Schlafes reguliert wird


- Störungen des zirkadianen Rhythmus durch Fernreisen oder Schichtarbeit erklären und eine Behandlngsmethode benennen
- Schlafstörungen benennen und erklären
- Epilepsie definieren und erklären
- erklären, wie sich das EEG bei einer epileptischen Episode darstellt 



EEG 

- Prinzip - Wiederholung von Max Vorlesung
- Anwendungen
- Messdiagramm
- Eigenschaften
- Arten von Wellen
- Beispiel: alpha Blockade

Zirkadiane Rhythmen

- Allgemeines zu Tag-Nacht-Rhythmus
- SCN und downstream Strukturen
- Molekularer Oszillator im SCN
- Einfluss von Licht
- Melatonin (?)

Schlaf

- Funktion
- Schlafphasen
- Schlafphasen im EEG
- Neuronale Kontrolle des Schlafes
    Uebergang Schlaf-Wachheit
        2-Prozess-Theorie
    Uebergang REM/non-REM
- Schlafstoerungen
    Sonia
    Studie Ärzte Schlafmangel
    Alter
    Benzos

Epilepsie



*********************************
*** Feinplanung
*********************************

ZNS II: EEG, Schlaf, Epilepsie

Nach  einer  kurzen  Rekapitulation  der  EEG-Enstehung  erläutern  die Studierenden  die  Charakteristika  wie  Grundrhythmus und  Frequenzbänder  des  Ruhe-EEG  und  dessen  Änderung  unter  physiologischen Bedingungen  (Schlaf,  Pharmakologie)  und  pathphysiologischen  Bedingungen (Epilepsie, Hirntoddiagnostik). Ferner  wird  die  Bedeutung evozierter Hirnaktivität mittels evozierter Potentiale für die Bedeutung klinisch-neurophysiologischer und kognitiver  Diagnostik erläutert und deren technisch-praktische Ableitung und Analyse in Grundzügen verstanden.  Schließlich  werden  die  Bedeutung,  Steuerung  und  Klassifikation von Schlaf (Stadien, REM, non-REM), dessen Diagnostik und die  zirkadiane Rhythmik  von  den  Studierenden  erläutert  sowie  die Grundlagen der Epilepsieentstehung verstanden.


*********************************
*** IMPP Katalog
*********************************

20.2.1 zirkadiane Periodik

zerebrale Generatoren zirkadianer Uhren, Wach-Schlafzyklus und seine Anpassung an den Tag-Nacht-Wechsel; Schlafperioden und Schlafmuster; Altersabhängigkeit der Schlafmuster; pathologische Schlafmuster; Schlafstadiendiagnostik mit EEG, EMG und Okulogramm; vegetative Funktionen in REM- und nREM-Schlaf; REM-Schlaf und Träumen; Hypothesen der neuronalen Schlafsteuerung

* Somnambulismus, Narkolepsie, Enuresis nocturna; Schlafstörungen bei Alkohol- und Drogenabusus; Jet-lag, Schichtarbeit

20.1.4 Analyse der Hirnrindenaktivität und tiefer liegender
Kerngebiete 

elektrophysiologische Analyse von Hirnrindenaktivität auf verschiedenen Ebenen: extra- und intrazelluläre Einzelzellableitungen, Summenaktivität; elektrische Phänomene, die dem EEG zugrunde liegen, Frequenzanalyse von EEG und MEG, Wellenbänder und ihre funktionelle Bedeutung, Synchronisation, kortikale Gleichspannungspotentiale, er- eignisbezogene Potentiale (Mittelungsmethoden), Messung der Hirndurchblutung mit Radioisotopen, (funktionelle) Magnetresonanztomographie (MRT, fMRT), transkranielle Magnetstimulation

* Veränderungen der elektrischen Hirnerscheinungen bei Erkrankungen und unter Einfluss von Pharmaka; EEG-Analyse von Bewusstseinsstörungen und Krampfanfällen; Hirntoddiagnostik






