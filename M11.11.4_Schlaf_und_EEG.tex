\documentclass{beamer}	
\mode<presentation>
 
\usepackage{pdfpages}
\usepackage{fancyvrb}
\usepackage{chemarr}

\usepackage{amsmath}		%% mathematics typesetting
\usepackage{amssymb}
 
\usepackage{epigraph}   %% nice setting of quotations

\usepackage{tabularx} %% allows to use row colours in tables

\usepackage{ulem}

\usepackage{booktabs}

\usepackage{siunitx} %% tpyeset SI units

\usepackage{CJKutf8} %% typeset Chinese characters

\usepackage{pdfpages}%% include pdfs


\usepackage{animate} %% show animated gifs

\DeclareMathAlphabet{\mathcalligra}{T1}{calligra}{m}{n}


% Color and Theme. Can be changed. However, this one's quite nice.
\usetheme{Madrid}
\definecolor{theme}{rgb}{0.84,0,0.21}
\usecolortheme[named=theme]{structure}


%%  Title information
\title[M11.11.4 EEG, Schlaf, Epilepsie]{M11.11.4 Zentrales Nervensystem 2: EEG, Schlaf und Epilepsie}
\author[melanie.stefan@medicalschool-berlin.de]{}
\institute[]{Prof. Melanie Stefan \\ melanie.stefan@medcialschool-berlin.de}
\date{SoSe 2022}
 

% Table of contents to pop up at the beginning of each section
\AtBeginSection[]
{
  \begin{frame}<beamer>
    \frametitle{Outline}
    \tableofcontents[currentsection,currentsubsection]
  \end{frame}
}
 
\beamertemplatenavigationsymbolsempty

\begin{document}


{ \usebackgroundtemplate{\includegraphics[width=1.2\paperwidth]{MSB_Titelseite.pdf}} 
\begin{frame}

 \maketitle 

$\,$\\[6cm] 


\end{frame} 
}



%% Hook: 

{ \usebackgroundtemplate{  \includegraphics[width=1.2\textwidth]{koala.jpg}    
}
\begin{frame}

\frametitle{Was, wenn Sie während dieser Vorlesung einschlafen?}


\end{frame}
 }
 
 
%% %% TLIA


\begin{frame}
\frametitle{In dieser Vorlesung geht es um \dots }

Schlaf, Epilepsie, und wie sie am EEG aussehen

\begin{center}
    \includegraphics[width=\textwidth]{Eeg_theta.png}    

\end{center}


\end{frame}

%% %% Learning Objectives
 
\begin{frame}

 \frametitle{Nach dieser Vorlesung sollten Sie folgendes können}



\begin{block}{Grundlagen:}
\begin{itemize}
\item
Funktionsweise und Eigenschaften des EEG erklären
\item
Frequenzbereiche des EEG erkennen, benennen, und erklären
\item
erklären, wie sich das EEG im Schlaf verändert
\item
interne und externe Regulierung des zirkadianen Rhythmus erklären
\item
die Funktionen des Schlafes angeben
\item
die Schlafphasen benennen und beschreiben
\item
erklären, wie der Übergang zwischen Schlaf und Wachheit reguliert wird
\item
erklären, wie der Übergang zwischen verschiedenen Phasen des Schlafes reguliert wird
\end{itemize}

\end{block}
\end{frame}


 
\begin{frame}

 \frametitle{Nach dieser Vorlesung sollten Sie folgendes können}


\begin{block}{Klinik:}
\begin{itemize}
\item
Störungen des zirkadianen Rhythmus durch Fernreisen oder Schichtarbeit erklären und eine Behandlungsmethode benennen
\item
Schlafstörungen benennen und erklären
\item
Epilepsie definieren und erklären
\item
erklären, wie sich das EEG bei einer epileptischen Episode darstellt 
\end{itemize}

\end{block}


\end{frame}










%% %% %% Main Body
 
 
\section{EEG} 

% - Prinzip - Wiederholung von Max Vorlesung
\begin{frame}{Wir erinnern uns \dots}

\begin{columns}[c]

\begin{column}{5cm}
\includegraphics[width=\textwidth]{EEG_cap.jpg}
\end{column}

\begin{column}{5cm}
\textcolor{theme}{Was wird beim EEG gemessen?} \\
\textcolor{theme}{Was ist die räumliche und zeitliche Auflösung?}
\end{column}

\end{columns}

    
\end{frame}

\begin{frame}{Wir erinnern uns \dots}

\begin{columns}[c]

\begin{column}{5cm}
\includegraphics[width=\textwidth]{EEG_cap.jpg}
\end{column}

\begin{column}{5cm}
Spannungsunterschiede auf der Kopfhaut - EPSP von vielen synchronen Neuronen im Cortex  \\[0.5 cm]

Auflösung: \(0.5-30\,\)Hz, Zentimeter-Bereich

\end{column}

\end{columns}

    
\end{frame}


%% Synchronizität

{ \usebackgroundtemplate{\includegraphics[width=\paperwidth]{synchronisation.pdf}} 
\begin{frame}

 
\end{frame} 
}
 
 
%
% - Anwendungen
\begin{frame}{Anwendungen des EEG}
    
\begin{itemize}
    \item 
    Diagnose und Behandlungskontrolle von Epilepsie und Charakterisierung von Anfällen
    \item 
Schlafüberwachung und Diagnose von Schlafstörungen, Unterscheidung von Schlafstadien
\item 
Anästhesietiefe
\item
Untersuchung von Gehirnfunktionen bei Personen, die sich subjektiv zu ihrem Bewusstseinszustand nicht äußern können
\item
Analyse von Bewusstlosigkeit, Koma, Feststellung des Hirntodes
\item
Neurodegenerative Erkrankungen
\item
Evozierte Potentiale – Störungen der Sinnesorgane
\item
Hirnforschung
\end{itemize}    

    
\end{frame}

% - Messdiagramm

% Fourier transformation

% - Eigenschaften
% - Arten von Wellen
% - Beispiel: alpha Blockade


\section{Zirkadiane Rhythmen}

% - Allgemeines zu Tag-Nacht-Rhythmus

\begin{frame}{Viele biologische Vorgänge sind nachts und tagsüber unterschiedlich aktiv}

%% picture: Mouse actogram
\begin{center}
\includegraphics[width=0.8\textwidth]{Mouse_actogram_1.png}    
\end{center}

\pause 
\textcolor{theme}{Um welche Art von Tier handelt es sich hier? Was würde passieren, wenn wir das Tier permanent von Sonnenlicht isolieren?}

    
\end{frame}

%% Interne Komponente Actogram 2
\begin{frame}{Der zirkadiane Rhythmus has eine interne Komponente}


\begin{center}
\includegraphics[width=0.8\textwidth]{Mouse_actogram_2.png}    
\end{center}


Die "innere Uhr" läuft weiter, braucht aber etwas mehr als 24 Stunden. 

    
\end{frame}

%% Externe Komponente Actogram 3
\begin{frame}{Der zirkadiane Rhythmus wird durch Tageslicht (Zeitgeber) synchronisiert}

\begin{center}
\includegraphics[width=0.8\textwidth]{Mouse_actogram_3.png}    
\end{center}

    
\end{frame}


% - SCN und downstream Strukturen


% - Molekularer Oszillator im SCN
% - Einfluss von Licht
% - Melatonin (?)

% Störungen des zirkadianen Rhythmus

\section{Schlaf}


%% Definition

\begin{frame}{Schlaf}

\begin{columns}[c]

\begin{column}{5cm}
\begin{center}
    \includegraphics[width=0.8\textwidth]{Schlafende.jpg}
\end{center}
\end{column}

\begin{column}{5cm}
Wiederkehrender, reversibler Zustand verringerten Bewusstseins bei Menschen und Tieren
\end{column}

\end{columns}
    
\end{frame}

%% Funktionen 

\begin{frame}{Funktionen des Schlafes}

\begin{block} {Energie und Metabolismus}      
    \begin{itemize}
        \item "Energiesparmodus"
        \item Abtransport von Stoffwechselprodukten aus dem Gehirn 
        \item Auffüllen von Glykogenspeichern
    \end{itemize}
    
    
\end{block}

\pause

    %% Immunsystem

\begin{block}{Immunsystem}
 \begin{itemize}
     \item 
     Verstärkte Immunantwort
 \end{itemize}
\end{block}

\pause
    
    %% Gedächtnis
\begin{block}{Kognitive Funktionen}
 \begin{itemize}
     \item 
     Synaptische Homöostase (Herunterregulierung von tagsüber potenzierten Synapsen
     \item
     Gedächtniskonsolidierung
 \end{itemize}
\end{block}

    
    
\end{frame}


% - Schlafphasen

\begin{frame}{Schlafphasen}
    
    
\end{frame}



% - Schlafphasen im EEG
% - Neuronale Kontrolle des Schlafes
%     Uebergang Schlaf-Wachheitu
%         2-Prozess-Theorie
%     Uebergang REM/non-REM
% - Schlafstoerungen
%     Sonia
%     Studie Ärzte Schlafmangel
%     Alteröl
%     Benzos

\section{Epilepsie}

% Definition, Arten
\begin{frame}{Formen von epilleptischen Anfällen}


\begin{itemize}
    \item 
Tonisch: anhaltende Muskelkontraktionen 
\item 
Klonisch: Muskelzuckungen
\item
Tonisch-Klonisch
\item
Atonisch: Verlust von Muskelaktivität
\item
Absence: Verlust der Aufmerksamkeit, subtile körperliche Anzeichen
\end{itemize}

Dauer eines Anfalls im Sekunden- bis Minutenbereich. 


\end{frame}


% Stages of a seizure
{ \usebackgroundtemplate{\includegraphics[width=\paperwidth]{tonisch_klonisch.pdf}} 
\begin{frame}

 
\end{frame} 
}
 



% EEG

% Diagnose

% Medication


%% %% %% %% Review



\begin{frame}

 \frametitle{Jetzt* sollten Sie folgendes können}



\begin{block}{Grundlagen:}
\begin{itemize}
\item
Funktionsweise und Eigenschaften des EEG erklären
\item
Frequenzbereiche des EEG erkennen, benennen, und erklären
\item
erklären, wie sich das EEG im Schlaf verändert
\item
interne und externe Regulierung des zirkadianen Rhythmus erklären
\item
die Funktionen des Schlafes angeben
\item
die Schlafphasen benennen und beschreiben
\item
erklären, wie der Übergang zwischen Schlaf und Wachheit reguliert wird
\item
erklären, wie der Übergang zwischen verschiedenen Phasen des Schlafes reguliert wird
\end{itemize}

\end{block}
\end{frame}


 
\begin{frame}

 \frametitle{Jetzt* sollten Sie folgendes können}


\begin{block}{Klinik:}
\begin{itemize}
\item
Störungen des zirkadianen Rhythmus durch Fernreisen oder Schichtarbeit erklären und eine Behandlungsmethode benennen
\item
Schlafstörungen benennen und erklären
\item
Epilepsie definieren und erklären
\item
erklären, wie sich das EEG bei einer epileptischen Episode darstellt 
\end{itemize}

\end{block}


\end{frame}



%% %% %% %% Feedbackhinweisblock

\begin{frame}
\frametitle{Danke für Ihr Feedback!}

\begin{columns}[c]

\begin{column}{6cm}
\begin{center}
 \includegraphics[width=\textwidth]{smilie_balloons.jpg}
\end{center}

\end{column}

\begin{column}{4cm}


\begin{center}
\includegraphics[width=\textwidth]{feedback_QR.png}
\end{center}
\end{column}


\end{columns}

\end{frame}



%% %% %% Bildnachweis
\begin{frame}
\frametitle{Bildnachweis}
\begin{tiny}

Teile dieser Vorlesung wurden übernommen von einer Vorlesung von Prof. Emanuel Busch,  Health and Medical University Potsdam. Wo nicht anders gekennzeichnet, stammen Abbildungen aus dieser Vorlesung.  


 
\begin{itemize}

\item
Actogramm einer Maus. Modifiziert, von Christian A. Hundahl,Jan Fahrenkrug, Anders Hay-Schmidt, Birgitte Georg, Birgitte Faltoft, Jens Hannibal, CC BY-SA 4.0 \url{https://creativecommons.org/licenses/by-sa/4.0}, via Wikimedia Commons

\item
EEG bei Epilspsie. CC BY-SA 2.0, \url{https://commons.wikimedia.org/w/index.php?curid=845554}

\item
Glykogen. By Mikael Häggström.When using this image in external works, it may be cited as:Häggström, Mikael (2014). &quot;Medical gallery of Mikael Häggström 2014&quot;. WikiJournal of Medicine 1 (2). DOI:10.15347/wjm/2014.008. ISSN 2002-4436. Public Domain.orBy Mikael Häggström, used with permission. - Own work (Original text: Own work by uploader, using following images:)Glucose (Public Domain license).Glycogenin structure (Public Domain license).Structure reference: Similar image on scientificpsychic.com --; Carbohydrates - Chemical Structure. By Antonio Zamora on May 27, 2009, Public Domain \url{https://commons.wikimedia.org/w/index.php?curid=6880883} 

%% all lectures
\item
Luftballons mit frohen und traurigen Smilies. Photo by \href{https://unsplash.com/@artbyhybrid?utm_source=unsplash&utm_medium=referral&utm_content=creditCopyText}{Hybrid} on \href{https://unsplash.com/s/photos/feedback?utm_source=unsplash&utm_medium=referral&utm_content=creditCopyText}{Unsplash}
%%%%%%%%%%%


\item
Mann mit EEG Haube. Von Thuglas in der Wikipedia auf Englisch - Übertragen aus en.wikipedia nach Commons durch Sreejithk2000 mithilfe des CommonsHelper., Gemeinfrei,\url{https://commons.wikimedia.org/w/index.php?curid=10827060}

\item
Schlafende Frau. Von Ilja Jefimowitsch Repin - Tretyakov Gallery, Moscow, Gemeinfrei, \url{https://commons.wikimedia.org/w/index.php?curid=60387757
}

\item
Schlafender Koala. Photo by \href{https://unsplash.com/@davidclode?utm_source=unsplash&utm_medium=referral&utm_content=creditCopyText}{David Clode} on \href{https://unsplash.com/s/photos/sleep?utm_source=unsplash&utm_medium=referral&utm_content=creditCopyText}{Unsplash}

\item
Schlafzyklus. Von Schlafgut - Eigenes Werk, CC BY-SA 3.0, \url{https://commons.wikimedia.org/w/index.php?curid=24139958} 


\item

Theta-Wellen. Von Hugo Gamboa - This is raw eeg. The signal was acquired in the Oz position processed with scipy and saved with matplolib., CC BY-SA 3.0, \url{https://commons.wikimedia.org/w/index.php?curid=473247}
  

\end{itemize}
\end{tiny}
\end{frame}









\end{document}

%%% Frequently used snippets

%% \begin{columns}[c]

%% \begin{column}{5cm}
%% \end{column}

%% \begin{column}{5cm}
%% \end{column}


%% \end{columns}




