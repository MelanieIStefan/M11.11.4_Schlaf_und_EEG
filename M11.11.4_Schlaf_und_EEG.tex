\documentclass{beamer}	
\mode<presentation>
 
\usepackage{pdfpages}
\usepackage{fancyvrb}
\usepackage{chemarr}

\usepackage{amsmath}		%% mathematics typesetting
\usepackage{amssymb}
 
\usepackage{epigraph}   %% nice setting of quotations

\usepackage{tabularx} %% allows to use row colours in tables

\usepackage{ulem}

\usepackage{booktabs}

\usepackage{siunitx} %% tpyeset SI units

\usepackage{CJKutf8} %% typeset Chinese characters

\usepackage{pdfpages}%% include pdfs


\usepackage{animate} %% show animated gifs

\DeclareMathAlphabet{\mathcalligra}{T1}{calligra}{m}{n}


% Color and Theme. Can be changed. However, this one's quite nice.
\usetheme{Madrid}
\definecolor{theme}{rgb}{0.84,0,0.21}
\usecolortheme[named=theme]{structure}


%%  Title information
\title[M11.11.4 EEG, Schlaf, Epilepsie]{M11.11.4 Zentrales Nervensystem 2: EEG, Schlaf und Epilepsie}
\author[melanie.stefan@medicalschool-berlin.de]{}
\institute[]{Prof. Melanie Stefan \\ melanie.stefan@medcialschool-berlin.de}
\date{SoSe 2022}
 

% Table of contents to pop up at the beginning of each section
\AtBeginSection[]
{
  \begin{frame}<beamer>
    \frametitle{Outline}
    \tableofcontents[currentsection,currentsubsection]
  \end{frame}
}
 
\beamertemplatenavigationsymbolsempty

\begin{document}


{ \usebackgroundtemplate{\includegraphics[width=1.2\paperwidth]{MSB_Titelseite.pdf}} 
\begin{frame}

 \maketitle 

$\,$\\[6cm] 


\end{frame} 
}



%% Hook: 

{ \usebackgroundtemplate{  \includegraphics[width=1.2\textwidth]{koala.jpg}    
}
\begin{frame}

\frametitle{Was, wenn Sie während dieser Vorlesung einschlafen?}


\end{frame}
 }
 
 
%% %% TLIA


\begin{frame}
\frametitle{In dieser Vorlesung geht es um \dots }

Schlaf, Epilepsie, und wie sie am EEG aussehen

\begin{center}
    \includegraphics[width=\textwidth]{Eeg_theta.png}    

\end{center}


\end{frame}

%% %% Learning Objectives
 
\begin{frame}

 \frametitle{Nach dieser Vorlesung sollten Sie folgendes können}



\begin{block}{Grundlagen:}
\begin{itemize}
\item
Funktionsweise und Eigenschaften des EEG erklären
\item
Frequenzbereiche des EEG erkennen, benennen, und erklären
\item
erklären, wie sich das EEG im Schlaf verändert
\item
interne und externe Regulierung des zirkadianen Rhythmus erklären
\item
die Funktionen des Schlafes angeben
\item
die Schlafphasen benennen und beschreiben
\item
erklären, wie der Übergang zwischen Schlaf und Wachheit reguliert wird
\item
erklären, wie der Übergang zwischen verschiedenen Phasen des Schlafes reguliert wird
\end{itemize}

\end{block}
\end{frame}


 
\begin{frame}

 \frametitle{Nach dieser Vorlesung sollten Sie folgendes können}


\begin{block}{Klinik:}
\begin{itemize}
\item
Störungen des zirkadianen Rhythmus durch Fernreisen oder Schichtarbeit erklären und eine Behandlungsmethode benennen
\item
Schlafstörungen benennen und erklären
\item
Epilepsie definieren und erklären
\item
erklären, wie sich das EEG bei einer epileptischen Episode darstellt 
\end{itemize}

\end{block}


\end{frame}










%% %% %% Main Body
 
 
\section{EEG} 

% - Prinzip - Wiederholung von Max Vorlesung
\begin{frame}{Wir erinnern uns \dots}

\begin{columns}[c]

 


\end{columns}

    
\end{frame}


% - Anwendungen
% - Messdiagramm

% Fourier transformation

% - Eigenschaften
% - Arten von Wellen
% - Beispiel: alpha Blockade

\section{Zirkadiane Rhythmen}


% - Allgemeines zu Tag-Nacht-Rhythmus
% - SCN und downstream Strukturen
% - Molekularer Oszillator im SCN
% - Einfluss von Licht
% - Melatonin (?)

\section{Schlaf}3 Jul 2022


%% Definition

\begin{frame}{Schlaf}

\begin{columns}[c]

\begin{column}{5cm}
\begin{center}
    \includegraphics[width=\textwidth]{Schlafende.jpg}
\end{center}
\end{column}

\begin{column}{5cm}
Wiederkehrender, reversibler Zustand verringerten Bewusstseins bei Menschen und Tieren
\end{column}

\end{columns}
    
\end{frame}

%% Funktionen 

\begin{frame}{Funktionen des Schlafes }

\begin{block} {Energie und Metabolismus}    
    \begin{itemize}
        \item "Energiesparmodus"
        \item Abtransport von Stoffwechselprodukten aus dem Gehirn 
        \item Auffüllen von Glykogenspeichern
    \end{itemize}
    
    %% Immunsystem
    
    %% Gedächtnis
    
\end{block}

    
    
\end{frame}


% - Schlafphasen
% - Schlafphasen im EEG
% - Neuronale Kontrolle des Schlafes
%     Uebergang Schlaf-Wachheitu
%         2-Prozess-Theorie
%     Uebergang REM/non-REM
% - Schlafstoerungen
%     Sonia
%     Studie Ärzte Schlafmangel
%     Alter
%     Benzos

\section{Epilepsie}



% Definition, Arten

% Stages of a seizure

% EEG

% Diagnose

% Medication


%% %% %% %% Review



\begin{frame}

 \frametitle{Jetzt* sollten Sie folgendes können}



\begin{block}{Grundlagen:}
\begin{itemize}
\item
Funktionsweise und Eigenschaften des EEG erklären
\item
Frequenzbereiche des EEG erkennen, benennen, und erklären
\item
erklären, wie sich das EEG im Schlaf verändert
\item
interne und externe Regulierung des zirkadianen Rhythmus erklären
\item
die Funktionen des Schlafes angeben
\item
die Schlafphasen benennen und beschreiben
\item
erklären, wie der Übergang zwischen Schlaf und Wachheit reguliert wird
\item
erklären, wie der Übergang zwischen verschiedenen Phasen des Schlafes reguliert wird
\end{itemize}

\end{block}
\end{frame}


 
\begin{frame}

 \frametitle{Jetzt* sollten Sie folgendes können}


\begin{block}{Klinik:}
\begin{itemize}
\item
Störungen des zirkadianen Rhythmus durch Fernreisen oder Schichtarbeit erklären und eine Behandlungsmethode benennen
\item
Schlafstörungen benennen und erklären
\item
Epilepsie definieren und erklären
\item
erklären, wie sich das EEG bei einer epileptischen Episode darstellt 
\end{itemize}

\end{block}


\end{frame}



%% %% %% %% Feedbackhinweisblock

\begin{frame}
\frametitle{Danke für Ihr Feedback!}

\begin{columns}[c]

\begin{column}{6cm}
\begin{center}
 \includegraphics[width=\textwidth]{smilie_balloons.jpg}
\end{center}

\end{column}

\begin{column}{4cm}


\begin{center}
\includegraphics[width=\textwidth]{feedback_QR.png}
\end{center}
\end{column}


\end{columns}

\end{frame}



%% %% %% Bildnachweis
\begin{frame}
\frametitle{Bildnachweis}

\begin{tiny}

% Teile dieser Vorlesung wurden übernommen von einer Vorlesung von Prof. Maike Glitsch, Medical School Hamburg. Wo nicht anders angegeben, stammen Abbildungen aus dieser Vorlesung.  Herzlichen Dank!


 
\begin{itemize}


%% all lectures
\item
Luftballons mit frohen und traurigen Smilies. Photo by \href{https://unsplash.com/@artbyhybrid?utm_source=unsplash&utm_medium=referral&utm_content=creditCopyText}{Hybrid} on \href{https://unsplash.com/s/photos/feedback?utm_source=unsplash&utm_medium=referral&utm_content=creditCopyText}{Unsplash}
%%%%%%%%%%%


\item
Mann mit EEG Haube. Von Thuglas in der Wikipedia auf Englisch - Übertragen aus en.wikipedia nach Commons durch Sreejithk2000 mithilfe des CommonsHelper., Gemeinfrei,\url{https://commons.wikimedia.org/w/index.php?curid=10827060}

\item
Schlafende Frau. Von Ilja Jefimowitsch Repin - Tretyakov Gallery, Moscow, Gemeinfrei, \url{https://commons.wikimedia.org/w/index.php?curid=60387757
}

\item
Schlafender Koala. Photo by \href{https://unsplash.com/@davidclode?utm_source=unsplash&utm_medium=referral&utm_content=creditCopyText}{David Clode} on \href{https://unsplash.com/s/photos/sleep?utm_source=unsplash&utm_medium=referral&utm_content=creditCopyText}{Unsplash}

\item

Theta-Wellen. Von Hugo Gamboa - This is raw eeg. The signal was acquired in the Oz position processed with scipy and saved with matplolib., CC BY-SA 3.0, \url{https://commons.wikimedia.org/w/index.php?curid=473247}
  

\end{itemize}
\end{tiny}
\end{frame}









\end{document}

%%% Frequently used snippets

%% \begin{columns}[c]

%% \begin{column}{5cm}
%% \end{column}

%% \begin{column}{5cm}
%% \end{column}


%% \end{columns}




